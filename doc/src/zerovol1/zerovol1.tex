\documentclass[llpt, twoside, a4paper]{book}
\usepackage[T1]{fontenc}
\usepackage{parskip}
\usepackage{listings}
\usepackage{fancyvrb}
\lstset{language=C, showspaces=false, breaklines=false}

\begin{document}

\title{\LARGE{Zero Operating System}
       \LARGE{Volume One, the Kernel}
       \LARGE{DRAFT VERSION 1}}
\author{Tuomo Petteri Ven�l�inen}
\date{\today}
\maketitle

\setcounter{secnumdepth}{3}
\setcounter{tocdepth}{4}
\setlength{\parindent}{0cm}

	Design and Implementation of the Zero Kernel.

	Copyright (C) 2012 Tuomo Petteri Ven�l�inen

\tableofcontents

\part{Preface}

\chapter{Goals}

	\textbf{POSIX}

	Zero kernel aims to implement a POSIX-compatible Unix-like system.
	POSIX-compliance will be provided at library level, where the base
	kernel's system call and other interfaces will provide the required
	functionality.

	\textbf{Software Development}

	As a software developer, I find it crucial that an operating system has
	good, modern tools for software development. I will eventually look into
	porting GNU and other tools such as the GNU C Compiler (GCC), the GNU
	Debugger (gdb), and the GNU Profiler (gprof), as well as experiment with
	new functionality such as graphical frontends for them.

	\textbf{Multimedia}

	Another goal of Zero is to provide good support for multimedia; both
	playback and production. Such functionality would include libraries and
	APIs for audio, video, still images, and so forth. One idea is to port
	software such as SDL early on to provide support for applications such
	as audio and video editors as well as games.

	\textbf{Research and Education}

	I see it as an important goal of the project to provide an open source
	system for use in research and education. The system should capture some
	of the original, elegant philosophy of early Unix. The kernel should be
	[relatively] easy to modify for targeted use. I have chosen a relatively
	liberal '2-clause BSD' license which projects such as the FreeBSD
	operating system use. I chose it over the GNU General Public License
	(GPL) for easier adoption to commercial projects such as smart phones.

	\textbf{Portability}

	The plan is to make the kernel portable to different platforms, and
	provide support for IA-32, X86-64, as well as ARM from early on.

\chapter{Overview}

\section{Multithread Kernel}

	Zero will be a multithread kernel, designed for multiprocessor and
	multicore systems from the ground up. Zero will have a few separate
	threads and processes for certain system functionality. For example, the
	page daemon might run as its own thread, where init will most likely be
	a separate process, spawning new processes as its children.

\section{Multiuser System}

	Zero will be a multitasking, multiprocessing kernel. Zero shall support
	virtual memory as a required part of a multiuser system to provide
	protection and separation on per-process basis.

\section{Networking}

	Even though networking is not one of the first things to do, I plan to
	support IPv4 and IPv6 with the BSD Unix sockets API.

\section{User Interfaces}

	Zero will have both command line and graphical interfaces. The graphical
	interface is going to be a network-enabled, event-based system; I will
	probably support the X Window System as well to let us run the myriad of
	existing applications [also known as clients] in circulation.

\section{C Library}

	The kernel will be distributed with other software such as a C library
	with support for functionality from standards like ISO C, POSIX, X/Open
	and other relevant standards.

	At the time of writing this, some machine dependent parts of the C
	library such as setjmp(), longjmp(), and alloca() are implemented for
	IA-32, X86-64, and ARM. This choice of platforms reflects most modern
	desktop and embedded systems, with plans to make it easy to port the
	kernel and accompanying software to other platforms as well.

\section{Zero Library}

	Zero implements functionality for both the kernel and user environment
	in a library called Zero. Currently, there's support for mutexes using
	atomic compare and swap; these mutexes tend to be faster than those
	implemented as a part of the POSIX Thread Library (pthread). I will see
	about implementing the ISO C11 API for multithread functionality.

\part{Zero Kernel}

\chapter{Build Environment}

\section{System Parameters}

	Some system parameters are declared for the programmers through
	\textbf{<zero/param.h>}. These declarations include sizes of certain
	types, pages, and cachelines. The following is the file
	\textbf{<zero/ia32/param.h>} which gets included on IA-32 systems.

	\textbf{<zero/param.h>}

\VerbatimInput{../../../usr/lib/zero/ia32/param.h}

\newpage

\section{Compiler Attributes}

	<zero/cdecl.h> declares attributes specific to the C compiler in use.
	Some of these attributes are specified in ways of both ISO C standards
	as well as compiler-specific ones which often predate standardisation by
	the ISO C Committee. The following is the file \textbf{<zero/cdecl.h>}.

	\textbf{<zero/cdecl.h>}

\VerbatimInput{../../../usr/lib/zero/cdecl.h}

\newpage

\chapter{Bootstrap}

\section{GRUB Support}

	\textbf{Multiboot}

	Zero uses the Grand Unified Boot Loader (GRUB) to start the system.
	Using GRUB requires a multiboot header; the following is the file
	\textbf{kern/unit/ia32/boot.h} which shows the multiboot header and
	parameters in C syntax (see \textbf{struct mboothdr}).

\newpage

	\textbf{kern/unit/ia32/boot.h}

\VerbatimInput{../../../kern/unit/ia32/boot.h}

	\textbf{TODO: use memory map provided by GRUB.}

\newpage

	\textbf{Bootstrap Code}

	The following is our bootstrap code utilising GRUB with a multiboot
	header.

	\textbf{kern/unit/ia32/boot.S}

\VerbatimInput{../../../kern/unit/ia32/boot.S}

\newpage

\section{System Initialisation}

	\textbf{Assembly Routines}

	The following code is \textbf{kern/unit/ia32/setup.S}, which implements
	assembly routines for system initialisation. I will refer to parts of
	this code later in the related subsections of this chapter.

	\textbf{kern/unit/ia32/setup.S}

\VerbatimInput{../../../kern/unit/ia32/setup.S}

\newpage

\subsection{Segmentation}

	\textbf{Kernel Segments}

	IA-32 architecture requires use of segmentation. Zero implements this
	in a simple way, relying on paging for protection and other tasks. The
	base segments are defined as follows.

\begin{tabular}{ | l | l | l | l | }
	\hline
	ID & Segment & Protection & Purpose \\
	\hline
	0  & NULL    & no access                & NULL/zero segment required by the CPU \\
	1  & TEXT    & read + execute           & kernel code segment \\
	2  & DATA    & read + write             & kernel data segment \\
	3  & STK     & read + write + grow-down & kernel stack \\
	4  & TSS     & read + write             & kernel task state segment \\
	5  & UTEXT   & read + execute + user    & user code segment \\
	6  & UDATA   & read + write             & user data segment \\
	7  & CPU     & read + write             & per-CPU data \\
	\hline
\end{tabular}

\subsection{Paging}

	\textbf{Page Structures}

	IA-32 uses two-level page structure; a single page for page directory
	whose entries point to page tables, the entries of which in turn point
	to pages.

	\textbf{Recursive Paging}

	I use Jolitz-style recursive paging, which is covered in depth in the
	chapter \textbf{Memory Management}.

\newpage

\subsection{Interrupt Initialisation}

	\textbf{Interrupt Vector}

	Interrupt vector, or \textbf{IDT} (interrupt descriptor table), sets up
	handler functions for interrupts. This table is initialised in
	\textbf{kern/unit/ia32/trap.c} and activated in
	\textbf{kern/unit/ia32/setup.S}.

	\textbf{kern/unit/ia32/trap.c}

\VerbatimInput{../../../kern/unit/x86/trap.c}

	\textbf{idtset()}

	Our interrupt vector is enabled with the \textbf{idtset} function in
	\textbf{kern/unit/ia32/setup.S} shown earlier in this chapter.

\newpage

\section{Multiprocessor Bootstrap}

	I'm working on multiprocessor scheduling and execution for my operating
	system project called Zero.

	The following is \textbf{currently unworking code} for our
	IA-32 multiprocessor startup sequence.
	\textbf{Please help me fix this	code}.

	\textbf{kern/unit/ia32/mpentry.S}

\VerbatimInput{../../../kern/unit/ia32/mpentry.S}

\newpage

\section{Linker Script}

	\textbf{Physical Memory}

	The linker script describes the physical memory layout of our loaded
	kernel image. We use GNU linker script syntax to achieve this goal. The
	following is the linker script for the IA-32 kernel. The segments are
	discussed in more detail in the \textbf{Memory Management} chapter.

	\textbf{kern/unit/ia32/kern.lds}

\VerbatimInput{../../../kern/unit/ia32/kern.lds}

\newpage

\chapter{Interrupt Management}

\section{General Information}

	\textbf{Interrupts by Nature}

	Interrupts are asynchronous events indicating things such as hardware
	and software errors as well as I/O (input and output) operations.

	\textbf{Hardware and Software}

	Interrupts can be triggered by hardware (e.g. user or device I/O
	request) as well as software (e.g. division by zero).

	\textbf{Terminology}

	Traditional Unix terminology calls internal interrupts, i.e. those
	occurring as the result of event internal to CPU, \textbf{traps}.
	External interrupts, i.e. requests from devices, are called
	\textbf{interrupts}. I chose to use more traditional PC terminology of
	\textbf{interrupt requests} (\textbf{IRQs}) for external interrupts.
	When necessary, I refer to traps and interrupt requests collectively as
	interrupts.

	\textbf{Traps vs. Interrupt Requests}

	A \textbf{noteworthy difference} between traps and interrupt
	requests is that the CPU \textbf{triggers interrupt handlers right off
	on traps}. However, with \textbf{interrupt requests},
	their \textbf{disposition may be postponed} if other higher priority
	interrupts are being dealt with.

	\textbf{Interrupt Service Routines}

	Interrupts are managed by hooking special handler routines, called
	interrupts service routines, to them. The mechanism to do this is CPU
	dependent and is described for the IA-32 platform later in this chapter.

\section{Special Interrupts}

\subsection{System Call Interrupt}

	\textbf{Library Support}

	System call interrupts are software generated ones used to request the
	kernel for services. Zero C library shall implement a library-level
	interface to this kernel functionality. The system call interface is
	described elsewhere in this book in a chapter called
	\textbf{System Call Interface}.

	\textbf{System Call Implementation}

	It is noteworthy that there exist other ways of implementing system
	calls; these ways tend to be used to speed system call processing up.
	On later IA-32 implementations, this may be done with the machine
	instructions \textbf{SYSENTER} and \textbf{SYSEXIT}.

	The 'slow' way of doing system calls with interrupts is implemented
	in the name of tradition as well as to support older hardware.

\subsection{Page Fault Exception}

	\textbf{Details}

	Page faults are interrupts of special interest as the means to
	implement kernel page daemon. A page fault is generated every time a
	page not in physical memory is accessed. The kernel can then do its
	magic. This exception pushes an error code with the fault address and a
	few flag bits (\textbf{system/user},
	\textbf{read/write}, and \textbf{present} flags). The kernel bases its
	[virtual] memory management based on this error code and its paging
	algorithm. Virtual memory is covered in more depth in the
	\textbf{Memory Management} chapter.

\section{Interrupt List}

	\textbf{Traps and IRQs}

	This section lists standard IA-32 traps, i.e. CPU exceptions as
	well as hardware IRQs (interrupt requests).

	Note that interrupt numbers \textbf{0x14 through 0x20} are
	\textbf{reserved}.

	\textbf{Notes}

\begin{itemize}
	\item{The \textbf{Page Fault} exception (interrupt) (0x0e) is of special
	interest to implementors of kernel-level page management}
	\item{Some interrupts push an error-code on stack, other's don't; I take
	care to take this in account in our interrupt handler/service
	routines}
	\item{The \textbf{NMI}, i.e. non maskable interrupt, is a result of a
	hardware failure (like a memory parity error), or watchdog timer which
	can be used to detect kernel lock-ups}
	\item{\textbf{Faults} leave the EIP point at the faulting instruction}
	\item{\textbf{Traps} leave the EIP point at the instruction right after
	the one that caused the interrupt}
    	\item{\textbf{Aborts} may not set the return instruction pointer right,
	so should be acted on by shutting the process down}
\end{itemize}

\begin{tabular}{ | l | l | l | l | l | }
	\hline
	Mnemonic & Number & Class & Error Code & Explanation \\
	\hline
	DE   & 0x00 & fault      &  no & Divide Error \\
	DB   & 0x01 & fault/trap &  no & Reserved \\
	NMI  & 0x02 & interrupt  &  no & Non Maskable Interrupt \\
	BP   & 0x03 & trap       &  no & Breakpoint \\
	OF   & 0x04 & trap       &  no & Overflow \\
	BR   & 0x05 & fault      &  no & BOUND Range Exceeded \\
	UD   & 0x06 & fault      &  no & Invalid (Undefined) Opcode \\
	NM   & 0x07 & fault      &  no & No Math Coprocessor \\
	DF   & 0x08 & fault      &  0  & Double Fault \\
	RES1 & 0x09 & fault      &  no & Coprocessor Segment Overrun (reserved) \\
	TS   & 0x0a & fault      & yes & invalid TSS (task state segment) \\
	NP   & 0x0b & fault      & yes & Segment Not Present \\
	SS   & 0x0c & fault      & yes & Stack-Segment Fault \\
	GP   & 0x0d & fault      & yes & General Protection \\
	PF   & 0x0e & fault      & yes & Page Fault \\
	RES2 & 0x0f & N/A        &  no & Intel Reserved \\
	MF   & 0x10 & fault      & yes & x87 FPU Floating-Point Error (Math Fault) \\
	AC   & 0x11 & fault      &  0  & Alignment Check \\
	MC   & 0x12 & abort      &  no & Machine Check \\
	XF   & 0x13 & fault      &  no & SIMD Exception \\
	\hline
\end{tabular}

\section{Signal Map}

	\textbf{Trap Signals}

	The table below describes mapping of traps to software signals to be
	delivered to the corresponding process.

\begin{tabular}{ | l | l | }
	\hline
	Interrupt & Signal \\
	\hline
	DE        & SIGFPE \\
	DB        & not mapped \\
	NMI       & not mapped \\
	BP        & SIGTRAP \\
	OF        & not mapped \\
	BR        & SIGBUS \\
	UD        & SIGILL \\
	NM        & SIGILL \\
	DF        & not mapped \\
	RES1      & not mapped \\
	TS        & not mapped \\
	NP        & SIGSEGV \\
	SS        & SIGSTKFLT \\
	GP        & SIGSEGV \\
	PF        & not mapped \\
	RES2      & not mapped \\
	MF        & SIGFPE \\
	AC        & SIBGUS \\
	MC        & SIGABRT \\
	XF        & SIGFPE \\
	\hline
\end{tabular}

\section{Interrupt Requests (IRQs)}

\subsection{IRQ Map}

	Interrupt controllers can be programmed to map IRQs as the system
	wishes; our kernel maps interrupt requests conventionally to interrupts
	\textbf{0x20 through 0x2f}.

\begin{tabular}{ | l | l | l | l | }
	\hline
	Interrupt & Function & Number    & Brief \\
	\hline
	0x20      & Timer           & 0  & interrupt timer \\
	0x21      & Keyboard        & 1  & keyboard interface \\
	0x22      & Cascade         & 2  & Tied to IRQs 8-15 \\
	0x23      & COM2/COM4       & 3  & serial ports \#2 and \#4 \\
	0x24      & COM1/COM3       & 4  & serial ports \#1 and \#3 \\
	0x25      & Printer Port 2  & 5  & parallel port \#2, often soundcard \\
	0x26      & Floppy Drive    & 6  & floppy drive interface \\
	0x27      & Printer Port 1  & 7  & parallel port \#1 \\
	0x28      & Real-Time Clock & 8  & clock \\
	0x29      & IRQ2 Substitute & 9  & \\
	0x2c      & Mouse           & 12 & PS/2 mouse interface \\
	0x2d      & FPU             & 13 & floating point coprocessor \\
	0x2e      & IDE Channel 0   & 14 & disk controller \#1 \\
	0x2f      & IDE Channel 1   & 15 & disk controller \#2 \\
	\hline
\end{tabular}

\chapter{Thread Scheduler}

\section{Scheduler Classes}

	Zero scheduler has several different scheduler classes encapsulated
	into the table below.

\begin{tabular}{ | l | l | l | }
	\hline
	Class & Characteristis & Example \\
	\hline
	RT     & real-time, fixed priority         & multimedia buffering \\
	USER   & interactive, short CPU bursts     & terminal session \\
	BATCH  & batch processing, CPU-intensive   & compiler instance \\
	IDLE   & executed when system is idle      & zeroing pages \\
	\hline
\end{tabular}

	\textbf{Interactive Tasks}

	The scheme above will be extended; at least some interactive tasks
	could be run for purposes such as dispatching events, using shorter-
	than-usual scheduler time slices. This might help making the system
	more responsive	to user interaction.

	\textbf{Interval Tasks}

	In addition to these, Zero runs certain services every few timer
	interrupts to achieve tasks such as high-speed screen synchronisation.
	These interval tasks should be fast to execute not to degrade speed of
	other threads.

\section{Thread Priorities}

	Zero supports 32 priorities per scheduler class, hence having a total
	of 128 run queues to pick threads to run from.

\section{Timer Interrupts}

	Interval tasks run using the standard PIT (programmable interrupt
	timer). Per-CPU thread scheduling is done with local APIC timer
	interrupts.

\newpage

\subsection{PIT/8253 Timer}

	The following header file is a part of our driver for the 8253 timer
	chip (also known as the PIT, short for programmable interrupt timer).

\VerbatimInput{../../../kern/unit/x86/pit.h}

\newpage

	The following code implements our PIT initialisation.

\VerbatimInput{../../../kern/unit/x86/pit.c}

\newpage

\subsection{Local APIC Timers}

\chapter{Memory Management}

\section{Segmentation}

	Zero mostly uses so-called 'flat memory model', where each segment
	maps the whole address space with access suitable for the use. The
	segments are set as follows.

	\textbf{Flat Memory Model}

\begin{tabular}{ | l | l | l | l | }
	\hline
	Segment & Base        & Limit              & Permissions \\
	\hline
	NULL    & 0x00000000  & 0x00000000         & none \\
	TEXT    & 0x00000000  & 0xffffffff         & RX \\
	DATA    & 0x00000000  & 0xffffffff         & RW \\
	TSS     & tss \#coreid & sizeof(struct tss) & RW \\
	UTEXT   & 0x00000000  & 0xffffffff         & URX \\
	DATA    & 0x00000000  & 0xffffffff         & URW \\
	CPU     & cpu \#coreid & sizeof(struct cpu) & RW \\
	\hline
\end{tabular}

	\textbf{Permissions}

\begin{itemize}
	\item{U stands for user permission}
	\item{R stands for read permission}
	\item{W stands for write permission}
	\item{X stands for execute permission}
\end{itemize}

\section{Virtual Memory}

\subsection{Page Structures}

	\textbf{IA-32}

	On IA-32 architecture, we use Jolitz-style recursive page directory.
	This page directory maps our 4-megabyte table of page tables, which in
	turn map the individual pages. This simplifies our page address
	calculations and so makes the pager run faster.

	\textbf{Recursive Page Directory}

	The page directory is declared in \textbf{kern/unit/ia32/boot.S}.
	It's just a single page of physical memory identity-mapped at the same
	virtual and physical address. A single page directory entry points to
	the page directory itself, creating a recursion when looking for page
	addresses.

	\textbf{Table of Page Tables}

	The page tables are located at 8 megabytes physical, identity mapped to
	8 megabytes virtual. Effectively, this table is indexed with the page
	directory index and page table index of addresses to fetch addresses of
	the corresponding physical pages.

\newpage

	\textbf{Diagram}

	Below you can see a simple visualisation of our page structure
	hierarchy.

\begin{Verbatim}
        pagedir         pagetab
        -------         -------
        -------         -------
    --->| #0  |-------->| PG0 |
    |   -------         -------
    |     ...             ...
    |     ...             ...
    |     ...             ...
    |   -------         -------
    |---| DIR |    ---->| ADR | \
        -------    |    -------  |
          ...      |    | ... |  --- last page of pagetab
        -------    |    -------  |
        |#1023|----|    | ADR | /
        -------         -------

        pagedir
        -------
	- 4096 bytes (one page)
        - 1024 page table entries pointing to pages in pagetab

        pagetab
        -------
	- 4 megabytes
        - flat table of page pointers
\end{Verbatim}

\subsection{Identity Maps}

	Identity maps are mapped with their virtual addresses equal to their
	physical ones.

\subsubsection{HICORE}

	The HICORE segment contains code and data for early kernel bootstrap
	and initialisation.

\subsubsection{MP}

	The MP segment contains application processor startup code for
	multiprocessor systems.

\subsubsection{DMA}

	The DMA segment contains DMA buffers for device I/O.

\subsubsection{PAGETAB}

	The PAGETAB segment contains a flat table of page tables (containing
	page pointers).

\subsubsection{KERNVIRT}

	The KERNVIRT segment is mapped at the virtual address 0xc0000000
	(3 gigabytes) and contains kernel TEXT, DATA, and BSS segments.

\subsubsection{Kernel Map}

\begin{tabular}{ | l | l | l | }
\hline
Address             & Segment  & Brief \\
\hline
0 M                 & LOCORE   & low 1 megabyte \\
MPSTKSIZE           & MPSTK0   & processor 0 kernel-mode stack \\
1 M                 & HICORE   & early kernel bootstrap \\
1 M + SIZEOF(.boot) & MP       & multiprocessor startup code \\
4 M                 & DMA      & DMA I/O buffers \\
8 M                 & PAGETAB  & table of page tables \\
0xc0000000          & KERNVIRT & kernel TEXT, DATA, and BSS segments \\
\hline
\end{tabular}

\subsection{Memory Interface}

	This header file declares parts of our virtual memory manager.

	\textbf{kern/unit/ia32/vm.h}

\VerbatimInput{../../../kern/unit/ia32/vm.h}

\chapter{I/O Operations}

\section{DMA Interface}

	This header file declares our DMA I/O interface.

	\textbf{kern/unit/x86/dma.h}

\VerbatimInput{../../../kern/unit/x86/dma.h}

\newpage

\chapter{System Call Interface}

\section{Process Interface}

\subsection{void halt(long flg);}

	The \textbf{halt} system call shuts the operating system down.

	\textbf{Arguments}

\begin{tabular}{ | l | l | }
	\textbf{flg} & Operation \\
	HALT\_REBOOT & restart system \\
\end{tabular}

\subsection{long exit(long val, long flg);}

	The \textbf{exit} system call exits a running process.

	\textbf{Arguments}

\begin{itemize}
	\item{the val argument will be passed as return value to a shell}
	\item{if the flg argument has \textbf{EXIT\_SYNC} bit set, synchronise
	stdin, stdout, and stderr}
	\item{if the flg argument has \textbf{EXIT\_PROFRES} bit set, print
	statistics on resource usage}
\end{itemize}

\subsection{void abort(void);}

	The \textbf{abort} system call exits a process erraneously and writes
	out a core dump. This dump can later be used to analyse program
	execution with a debugger.

\subsection{long fork(long flg);}

	The \textbf{fork} system call creates a copy of the running process and
	starts executing it.

	\textbf{Arguments}

\begin{itemize}
	\item{if the flg argument has \textbf{FORK\_COPYONWR} bit set,
	don't copy parent address space before pages are written on}
	\item{if the flg argument has \textbf{FORK\_SHARMEM} bit set,
	share address space with parent (a'la Unix vfork())}
\end{itemize}

\subsection{long exec(char *file, char *argv[], char *env[]);}

	The \textbf{exec} system call executes \textbf{file}, passing it the
	runtime argument strings in \textbf{argv} and environment in
	\textbf{env}.

	\textbf{Arguments}

	Both \textbf{argv} and \textbf{env} need to be terminated with NULL
	entries.

\subsection{long throp(long cmd, long parm, void *arg);}

	The \textbf{throp} system call provides control of running threads,
	creation of new threads, and other thread functionality.

	\textbf{Arguments}

\begin{tabular}{ | l | l | }
	\hline
	\textbf{cmd} & Brief \\
	THR\_NEW     & create new thread \\
	THR\_JOIN    & join running thread (wait for exit) \\
	THR\_DETACH  & detach running thread \\
	THR\_EXIT    & exit thread \\
	THR\_MTXOP   & mutex operation \\
	THR\_CLEANOP & cleanup; pop and execute cleanup functions \\
	THR\_KEYOP   & create or delete thread \\
	THR\_CONDOP  & condition operations; signal, broadcast, ... \\
	THR\_SYSOP   & system attribute settings \\
	THR\_STKOP   & stack operation \\
	THR\_RTOP    & realtime thread settings \\
	THR\_SETATTR & set thread attributes \\
	\hline
\end{tabular}

\subsection{long pctl(long cmd, long parm, void *arg);}

	The \textbf{pctl} system call provides some process control operations.

	\textbf{Arguments}

\begin{tabular}{ | l | l | }
	\hline
	\textbf{cmd}    & Operation \\
	PROC\_WAIT      & wait for process termination \\
	PROC\_USLEEP    & sleep for a given number of microseconds \\
	PROC\_NANOSLEEP & sleep for a given number of nanoseconds \\
\end{tabular}

	\textbf{PROC\_WAIT}

\begin{tabular}{ | l | l | }
	\hline
	\textbf{parm} & Operation \\
	\hline
	PROC\_WAITPID & wait for given process to terminate \\
	PROC\_WAITCLD & wait for a child process in a group to terminate \\
	PROC\_WAITGRP & wait for a child process in caller's group to terminate \\
	PROC\_WAITANY & wait for any child process to terminate \\
	\hline
\end{tabular}

\subsection{long sigop(long pid, long cmd, void *arg);}

	\textbf{Arguments}

\begin{tabular}{ | l | l | }
	\hline
	\textbf{cmd} & Operation \\
	\hline
	SIG\_WAIT    & wait for a signal; pause() \\
	SIG\_SETFUNC & set signal handler; signal(), sigaction() \\
	SIG\_SETMASK & set signal mask; sigsetmask() \\
	SIG\_SEND    & send signal to process; raise(), ... \\
	SIG\_SETSTK  & configure signal stack; sigaltstack() \\
	SIG\_SUSPEND & suspend until signal received; sigsuspend() \\
	\hline
\end{tabular}

	\textbf{SIG\_SETFUNC}

	Set signal disposition.

\begin{tabular}{ | l | l | }
	\hline
	\textbf{arg} & Operation \\
	\hline
	SIG\_DEFAULT & initialise default signal disposition \\
	SIG\_IGNORE  & ignore signal \\
	\hline
\end{tabular}

	\textbf{SIG\_SEND}

	Send signal to \textbf{pid}.

\begin{tabular}{ | l | l | }
	\hline
	\textbf{pid} & Operation \\
	\hline
	SIG\_SELF    & send signal to self \\
	SIG\_CLD     & send signal to child processes \\
	SIG\_PGRP    & send signal to process group \\
	SIG\_PROPCLD & propagate signal to child processes \\
	SIG\_PROPGRP & propagate signal to process group \\
	\hline
\end{tabular}

	\textbf{SIG\_PAUSE}

	Suspend process. Flag bits are passed in \textbf{arg}.

\begin{tabular}{ | l | l | }
	\hline
	\textbf{arg} bit & Operation \\
	\hline
	SIG\_EXIT & exit process on signal \\
	SIG\_DUMP & dump core on signal \\
	\hline
\end{tabular}

\section{Memory Interface}

\subsection{long brk(void *adr);}

	The \textbf{brk} system call adjusts process break, i.e. the current top
	of the project's heap.

	\textbf{Arguments}

\begin{itemize}
	\item{\textbf{adr} is the new break}
\end{itemize}

\subsection{void *map(long desc, long flg, struct memreg *reg);}

	\textbf{struct memreg}

\begin{Verbatim}
    struct memreg {
        void *base;
        long  ofs;
        long  len;
        long  perm;
    };
\end{Verbatim}

	\textbf{Arguments}

\begin{tabular}{ | l | l | }
	\hline
	\textbf{flg} & Operation \\
	\hline
	MAP\_FILE       & map file (or /dev/zero for anonymous memory \\
	MAP\_ANON       & map anonymous memory \\
	MAP\_NORMAL     & default map behavior \\
	MAP\_SEQUENTIAL & sequential access \\
	MAP\_RANDOM     & non-sequential access \\
	MAP\_WILLNEED   & needed in the near future \\
	MAP\_DONTNEED   & not needed soon \\
	MAP\_DONTFORK   & don't share with forked child processes \\
	MAP\_HASSEM     & map region with semaphore values \\
	MAP\_BUFCACHE   & cache I/O block using kernel mechanism \\
	\hline
\end{tabular}

\subsection{long umap(void *adr, size\_t size);}

	The \textbf{umap} system call unmaps file or anonymous memory.

\subsection{long mhint(void *adr, long flg, struct memreg *arg);}

	The \textbf{mhint} system call is used to hint the kernel about the type
	of use for mapped regions.

	\textbf{Arguments}

	For values for the \textbf{flg} argument, see \textbf{map} earlier in
	this chapter.

\section{Shared Memory Interface}

\subsection{long shmget(long key, size\_t size, long flg);}

	The \textbf{shmget} system call returns the shared memory identifier
	associated with \textbf{key}.

\subsection{void *shmat(long id, void *adr, long flg);}

	The \textbf{shmat} system call attaches shared memory segment
	\textbf{id} to the address space of the calling process.

	\textbf{Arguments}

\begin{itemize}
	\item{if \textbf{adr} is nonzero, it will be used as the base address
	for the attached segment; otherwise, a new virtual region shall be
	allocated}
	\item{\textbf{TODO: flg argument?}}
\end{itemize}

\subsection{long shmdt(void *adr);}

	The \textbf{shmdt} system call detaches a shared memory segment attached
	at \textbf{adr} in the calling process's address space.

\subsection{long shmctl(long id, long cmd, void *arg);}

	The \textbf{shmctl} system call is used for controllling attributes of
	shared memory segments.

	\textbf{TODO}

\part{Base Drivers}

\chapter{VGA Text Consoles}

	VGA text console is the base user interface for Zero before the advent
	of graphics drivers. There's always a place for separate consoles as
	command line interfaces, to display console messages from system, and
	so forth.

	VGA color buffer is located at \textbf{0xb8000} (under one megabyte) and
	identity-mapped to the same region in kernel virtual address space.
	Hence drawing text becomes simple writing of character + attribute
	values into this memory region.

\newpage

\section{VGA Text Interface}

	This header file declares our VGA text interface.

\textbf{kern/io/drv/pc/vga.h}

\VerbatimInput{../../../kern/io/drv/pc/vga.h}

\newpage

\section{VGA Console Interface}

	The following is our VGA text console driver.

\textbf{kern/io/drv/pc/vga.c}

\VerbatimInput{../../../kern/io/drv/pc/vga.c}

\chapter{PS/2 Keyboard and Mouse}

\section{Keyboard Driver}

	Here is a source file for our PS/2-connector or emulated USB keyboards.

\textbf{kern/io/drv/pc/ps2/kbd.c}

\VerbatimInput{../../../kern/io/drv/pc/ps2/kbd.c}

\newpage

\section{Mouse Driver}

	Here is a source file for our PS/2-connector mice.

\textbf{kern/io/drv/pc/ps2/mouse.c}

\VerbatimInput{../../../kern/io/drv/pc/ps2/mouse.c}

\chapter{AC97 Audio Interface}

	I chose to develop drivers for AC97 because it's not only very
	wide-spread, but also well documented.

	\textbf{TODO: extend this chapter}

\appendix

\chapter{Profiler Tools}

	Zero has simple tools for timing code execution both in
	microsecond-resolution wall clock time as well as CPU clock cycles.

	This chapter shows the implementation of these profiling tools as well
	as examples of their use.

\newpage

\section{Wall Clock Profiler}

\VerbatimInput{../../../usr/lib/zero/prof.h}

\newpage

\section{Cycle Profiler}

\VerbatimInput{../../../usr/lib/zero/ia32/prof.h}

\section{Examples}

\newpage

\subsection{Wall Clock Profiler}

\VerbatimInput{wallprof.c}

\newpage

\subsection{Cycle Profiler}

\VerbatimInput{wallprof.c}

\end{document}

