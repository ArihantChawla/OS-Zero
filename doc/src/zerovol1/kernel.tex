\documentclass[twoside, a4paper]{book}
\usepackage[T1]{fontenc}
\normalsize
\usepackage{listings}
\usepackage{parskip}
\usepackage{verbatim}
\usepackage{fancyvrb}
\usepackage{hyperref}
\lstset{language=C, showspaces=false, breaklines=false}
\hypersetup{backref, colorlinks=true, linkcolor=blue}

\begin{document}

\title{\LARGE{Zero Kernel} \\
       \large{Design and Implementation} \\
       \large{DRAFT 1, Revision 7}}
\author{Tuomo Petteri Ven�l�inen}
\date{\today}
\maketitle

\setcounter{secnumdepth}{3}
\setcounter{tocdepth}{4}
\setlength{\parindent}{0cm}

\part{Overview}

\tableofcontents

\chapter{Preface}

\section{Acknowledgements}

\section{History}

\chapter{System Concepts}

\section{Terminology}

	\textbf{Trap}

	A trap is a hardware- or software generated event. Other names for traps
	include \textbf{interrupts}, \textbf{exceptions}, \textbf{faults}, and
	\textbf{aborts}. As an example, keyboard and mouse input may generate
	interrupts to be handled by interrupt service routines (\textbf{ISR}s).

	\textbf{Process}

	A process is a running instance of a program. A process may consist of
	several threads. Threads of a process share the same address space, but
	have individual execution stacks.

	\textbf{Thread}

	Threads are the basic execution unit of programs. To utilise
	multiprocessor-parallelism, a program may consist of several threads
	of execution, effectively letting it do several computation and	I/O
	operations at the same time.

	\textbf{Task}

	In the context of Zero, the term task is synonymous the term process.

	\textbf{Interval Task}

	Short-lived, frequent tasks such as audio and video buffer
	synchronisation are scheduled with priority higher than normal tasks.
	It is likely such tasks should be given a slice of time shorter than
	other threads.

	\textbf{Virtual Memory}

	Virtual memory is a technique to map physical memory to per-process
	address space. The processes see their address spaces as if they were in
	total control of the machine. These per-process address spaces are
	protected from being tampered by other processes. Address translations
	are hardware-level (the kernel mimics them, though), so virtual memory
	is transparent to application, and most of the time, even kernel
	programmers.

	\textbf{Page}

	A page is the base unit of memory management. Hardware protection works
	on per-page basis. Page attributes include read-, write-, and
	execute-permissions. For typical programs, the text (code) pages would
	have read- and execute-permissions, wherease the program [initialised]
	data pages as well as [unitialised] BSS-segment pages would have read-
	and write- but not execute-permissions. This approach facilitates
	protection against overwriting code and against trying to execute code
	from the data and BSS-segments.

	\textbf{Segment}

	Processes typically consist of several segments. These include a text
	segment for [read-only] code, a data segment for initialised global
	data, a BSS-segment for uninitialised global data, and different
	debugging-related segments. Note that the BSS-segment is runtime-
	allocated, whereas the data segment is read from the binary image. Also
	Note that in a different context, segments are used to refer to hardware
	memory management.

	\textbf{Buffer}

	Buffers are used to avoid excess I/O system calls when reading and
	writing to and from I/O devices. This memory is allocated from a
	separate buffer cache, which can be either static or dynamic size.
	Buffer pages are 'wired' to memory; they will never get paged out to
	disk or other external devices, but instead buffer eviction leads to
	writing the buffer to a location on some device; typically a disk.

	\textbf{Event}

	Events are a means for the kernel and user processes to communicate with
	each other. As an example, user keyboard input needs to be encoded to
	a well known format (Unicode or in the case of a terminal, ISO 8859-1
	or UTF-8 characters) and then dispatched to the event queues of the
	listening processes. Other system events include creation and destroyal
	of files and directories.

	Zero schedules certain events on timers separate from the timer
	interrupt used for scheduling threads. At the time of such an event, it
	is dispatched to the event queues of [registered] listening processes;
	if an event handler thread has been set, it will be given a short time
	slice of the event timer. Event time slices should be shorter than
	scheduler time slices to not interfere with the rest of the system too
	much.

\chapter{System Interface}

\section{UNIX Features}

	\textbf{Concepts}

	Zero is influenced and inspired by AT\&T and BSD UNIX systems. As I
	think many of the ideas in these operating systems have stood the test
	of time	nicely, it feels natural to base Zero on some well-known and
	thoroughly-tested concepts.

	\textbf{Nodes}

	Nodes are similar with UNIX 'file' descriptors. All I/O objects,
	lock structures needed for IPC and multithreading, as well as other
	types of data structures are called nodes, collectively. Their [64-bit]
	IDs are typically per-host kernel \textbf{memory addresses} (pointer
	values) for kernel \textbf{descriptor data structures}.

\section{POSIX Features}

	\textbf{Threads}

	Perhaps the most notable POSIX-influenced feature in Zero kernel is
	threads. POSIX- and C11-threads can be thought of as light-weight
	'processes' sharing the parent process address space but having unique
	execution stacks. Threads facilitate doing several operations at the
	same time, which makes into better utilisation of today's multicore-
	and multiprocessor-systems.

\section{Zero Features}

	\textbf{Events}

	Possibly the most notable extension to traditional UNIX-like designs in
	Zero is the event interface. Events are interprocess communication
	messages between kernel and user processes. Events are used to notify
	of user device (keyboard, mouse/pointer) input, filesystem actions such
	as removal and destroyal of files or directories, as well as to
	communicate remote procedure calls and data between two processes
	(possibly on different hosts).

	Events are communicated using message passing; the fastest transport
	available is chosen to deliver messages from a process to another one;
	in a scenario like a local display connection, messages can be passed
	by using a shared memory segment mapped to the address spaces of both
	the kernel and the desktop server.

\part{Basic Kernel}

\chapter{Kernel Layout}

	\textbf{Monolithic Kernel}

	Zero is a traditional, monolithic kernel. It consists of several parts,
	some of which are highlighted below.

\begin{tabular}{ | l | l | }
	\hline
	Module        & Operation \\
	\textbf{tmr}  & hardware timer interface \\
	\textbf{thr}  & thread scheduler \\
	\textbf{vm}   & virtual memory manager \\
	\textbf{page} & page daemon \\
	\textbf{mem}  & kernel memory allocator \\
	\textbf{io}   & I/O primitives \\
	\textbf{buf}  & block/buffer cache management \\
	\hline
\end{tabular}

	The code modules above will be discussed in-depth in the later parts of
	this book.

\chapter{Kernel Environment}

	This chapter describes the kernel-mode execution environment.
	Hardware-specific things are described for the IA-32 and X86-64
	architectures.

\section{Processor Support}

\subsection{Thread Scheduler}

\subsubsection{Thread Data Structure}

\subsection{Interrupt Vector}

	The interrupt vector is an array of interrupt descriptors. The
	descriptors contain interrupt service routine base address for the
	function to be called to handle the interrupt.

\subsubsection{Interrupt Descriptors}

	Entries in the interrupt vector, i.e. interrupt descriptor table
	(\textbf{IDT}), are called interrupt descriptors. These descriptors,
	whereas a bit hairy format-wise, consist of interrupt service address,
	protection ring (system or user), and certain other attribute flags.

\section{Memory}

\subsection{Overview}

\subsection{Segment Descriptor Tables}

\subsubsection{Segment Descriptors}

\subsection{Paging Data Structures}

\subsubsection{Page Directory Entry}

\subsubsection{Page Table Entry}

\subsubsection{Page Directory}

\subsubsection{Page Tables}

\subsection{Page Daemon}

\subsubsection{Page Replacement Algorithm}

\section{User Environment}

\subsection{Memory Map}

\begin{tabular}{ | l | l | l | }
	\hline
	Segment & Brief                 & Parameters \\
	\hline
	stack   & process stack         & read, write, grow-down \\
	\hline
	map     & memory-mapped regions & read, write \\
	\hline
	heap    & process heap (sbrk()) & read, write \\
	\hline
	bss     & uninitialised data    & read, write, allocate \\
	\hline
	data    & initialised data      & read, write \\
	\hline
	text    & process code          & read, execute \\
	\hline
\end{tabular}

	\textbf{Notes}

\begin{itemize}
	\item{memory regions are shown from highest to lowest address}
	\item{the stack segment grows downwards in memory}
	\item{the BSS segment is allocated at run-time}
\end{itemize}

\begin{itemize}
	\item{segments are shown in descending address order}
\end{itemize}

\chapter{IA-32 Implementation}

\end{document}

\appendix

\chapter{Notes on Tuning Code}

\section{Preface}

	This document demonstrates low-level code optimization techniques as
	well as simple data structures useful for high-performance data
	processing. The goal is to introduce the reader to some aspects of the
	magic of tuning code.

	Compilers massage code, people tune code. There is a difference.
	The best compiler cannot undo a bad design. Tuning code may not be
	rocket science, but it's definitely art and lots of fun.

\section{KISS Principle}

	Keep it simple, stupid/silly. The philosophy behind tuning code. CPUs
	are reasonably simple and good at doing simple things. The less you make
	them do, the faster the code will run.

	The fact that algorithmic optimizations, i.e. picking efficient
	algorithms as the first part of optimization, won't be given much
	attention in this document. Instead, I will concentrate on lower-level
	optimizations such as data organization.

	I'm of the belief that tuning code is mostly about knowing the machine.
	Therefore, I will discuss CPU internals on a generic level in ways that
	should be widely applicable. You will see that there's more to efficient
	code than first meets the eye.

\section{C Language}

	The C language is one of the biggest contributions to software
	technology. C is basically as low-level as you can get without resorting
	to assembly. C has been given the nickname of "pseudo-assembly". The
	basic CPU environment of registers, memory, and I/O has been abstracted,
	but thinly. If this abstraction gets too much on your way, you can
	usually use inline assembly, i.e. mix assembly code with C.

	You can, most of the time, think of variables as registers. Most C
	programs use pointers quite a bit; they can be considered memory
	addresses. In short, C programmers are dealing with what the CPU is;
	operations on and between registers and memory.

\subsection{C Integer Operations}

	The following tables contain relative speeds of some C operations; note
	that the actual timings vary from platform to platform, so perhaps the
	most important thing concerning integral arithmetics is the slow
	execution of division and modulus operations.

	\textbf{Arithmetic Operations}

	\textbf{TODO: replace with new data, include shift operations}

\begin{verbatim}
	op	  ticks/op   microseconds    iterations
	--	  --------   ------------    ----------
	++        6          17              1024
	--        6          18              1024
	+         6          17              1024
	-         6          18              1024
	*         8          21              1024
	/         12         32              1024
	\%        36         86              1024

	logical ops
	op      tick/op usecs   count
	--      ------- -----   -----
	&       2       8       1024
	|       2       8       1024
	^       2       9       1024
	~       2       8       1024
	!       4       12      1024
\end{verbatim}

	\textbf{Notes}

	The speed measurements above were run on a single P3-450 CPU. Note that
	whereas most arithmetic operations are about equal in speed, integer
	division is a bit slower and modulus is very slow. Therefore,
	it's worth knowing that for integral x and power-of-two value p2,

\begin{verbatim}
	y= x % p2;

        is equivalent with

        y = x & (p2 - 1);
\end{verbatim}

	To find out if x is a power of two, you can use

\begin{verbatim}
	#define powerof2(x) (!((x) & ((x) - 1)))
\end{verbatim}

	In other words, you can replace the very slow modulus with very fast
	logical operations. Note how logical operations run quite a bit faster
	than arithmetic. ;)

\section{C Floating Point Operations}


\begin{verbatim}
	float
	-----
	op      tick/op usecs   count
	--      ------- -----   -----
	+       9       23      1024
	-       9       24      1024
	*       109     250     1024
	/       124     284     1024

	double
	------
	op      tick/op usecs   count
	--      ------- -----   -----
	+       9       25      1024
	-       9       25      1024
	*       51      118     1024
	/       125     286     1024

	long double
	-----------
	op      tick/op usecs   count
	--      ------- -----   -----
	+       11      29      1024
	-       12      30      1024
	*       14      35      1024
	/       40      95      1024
\end{verbatim}

	\textbf{Notes}

	As is the case with integer math, addition and subtraction are
	reasonably fast. However, division is very slow.

	An interesting thing to notice is that added precision may even speed
	things up. :)

\section{Computer Architecture}

\subsection{Von Neumann Machine}

	A Von Neumann machine consists of three parts - CPU, memory, and I/O
	facilities.

	\textbf{Dies and Cores}

	A core is a CPU execution unit, basically a set of hardware registers
	and mathematical units that operate on them.

	Multicore systems have several cores on the same die, i.e. the same
	package. So called SMP (symmetric multiprocessor) systems are more
	traditional. They consist of several dies with one core on each.

	\textbf{Conclusions}

	Most of the time, the programmer need not be concerned about the
	hardware details. It may, however, pay back to implement CPU-intensive
	applications using multiple threads in the hopes the system might be
	capable of executing several of the threads simultaneously.

\subsection{ALU}

	ALU, arithmetic-logical unit, is a central part of a CPU. Implementation
	is, naturally, architecture-dependent, but the unit is used for
	reasonably simple arithmetic and logical operations.

\subsection{FPU}

	FPU, floating point unit, is used for floating-point calculations.
	There may, for example, be machine instructions for operations such as
	sine and cosine.

	Even though the days of slow floating-point operations may be history,
	there's still a point in not using floating-point to minimize the number	of registers that need to be saved on task switches.

	As a special case, I'm trying to keep my kernel all-integer in order to
	speed up in-kernel context switches.

\subsection{MMU}

	MMU, memory management unit, is a part of a CPU that handles [kernel-
	level] virtual memory. Virtual addresses can be mapped to arbitrary
	physical addresses to facilitate paging data in and out transparently
	to the programmer. MMUs implement schemes to protect memory from
	unwanted access.

	As a fast form of IPC (interprocess communications), shared memory
	deserves to be mentioned. Shared memory can be implemented by mapping
	pieces of physical memory (RAM) to address spaces of several processes.

\subsection{Prefetch Queue}

	Prefetch queues are used to read chunks of instruction data at a time.
	It's a good idea not to use many branching constructs, i.e. jump around 
	in code, to keep the CPU from not having to flush its prefetch queue
	often.

\subsection{Branch Prediction}

	CPUs may implement branch prediction logic to try to prefetch code to
	be executed into CPU instruction cache. Branches should be avoided where
	possible to avoid pipeline stalls (slows execution speed down) and when
	they are necessary, it's good form to put the most likely branches to be
	taken first. In some cases, excess branches can be avoided either with
	switch/case, or with function pointer tables.

\subsection{Pipeline}

	CPU pipelines are used to split instruction execution into several
	phases and try to execute different phases of adjacent instructions in
	parallel.

	A common technique to schedule instructions is to avoid data
	dependencies in adjacent operations; targets of an operation shouldn't
	be sources for the next one. This will reduce pipeline stalls and keep
	the CPU running code faster.

\section{Basic Techniques}

\subsection{Inlining}

	Inlining is a technique to duplicate pieces of code in order to avoid
	function calls. Function calls cause overhead because register state
	needs to be saved and restored; As this overhead shouldn't be long on
	current computers, inlining should probably only be used for small
	operations and/or left to the compiler to do.

	To reduce instruction cache misses, it's probably better to avoid code-
	duplication from excess inlining.

	If you can deal with the so-called debugging possibilities, macros are
	perhaps the best and definitely the most portable way to inline code.

\subsection{Loop Unrolling}

	Unrolling a loop means executing several loop iterations in one actual
	loop iteration. This is done to avoid looping overhead, e.g. updating
	index variables and pipeline stalls.

\subsection{Constants}

	If you see a constant used repetitively in code, it may be a good idea
	to assign it to a variable to hint the compiler to store it into a
	register. This way, instructions may become shorter as there is no need
	to encode [big] constant values into instruction.

	It may be fruitful to make constants as small as possible; the compiler
	should realize to know this, but I find code more appealing to read when
	hex constants of 2, 4, 8, or 16 digits are used to represent 8-bit,
	16-bit, 32-bit, and 64-bit values respectively.

\subsection{Memory Access}

	In short, memory access should be avoided. When you need to do it, try		to avoid cache misses and keep your data aligned.

\subsection{Alignment}

	As a rule of thumb, align words to the boundaries of their size, i.e.
	where the address is an exact multiple of the size.

	CPUs read memory a fixed number of bytes at a time. This means memory
	operands are fetched a given number of bytes at a time; you should
	minimize this number.

\subsection{Cache}

	A bit higher, CPUs read a cacheline's worth of data at a time. To avoid
	cache misses and so slow access to main physical memory, you should
	keep your data on as few cachelines as possible; in structures, try to
	keep related values close to each other.

\section{Simple Data Structures}

\subsection{Bitmaps}

	Bitmaps are useful for storing simple state information. For example,
	if you really only have to differentiate two states from each other,
	say free and busy, it would be a big overhead to store boolean values
	into words.

\subsection{Lookup Tables}

	If, for example, property determination gets complex, especially if it
	leads to lots of branching, it may be feasible to use lookup tables to
	store operation values that can be accessed in constant time; note,
	however, that there's a memory access which may be a slow task.

	As a useful trick, long switch statements can often be replaced byy a
	short piece of code to index a function pointer tables with the 'case'
	values. This way, the execution time will be constant.

\subsection{Hash Tables}

	The idea of hash tables is to map given keys to indices into a [fixed-
	size] table. Key collisions can be dealt with simple linked lists.
	Picking a proper hash function for calculating keys, e.g. from character
	strings, should ensure that the lists (chains) don't get long so lookups
	will run fast. Note, though, that hash tables are practically impossible
	to sort without using other data structures to help. Hash tables may use
	lots of memory; pay attention to the table size.

\subsection{Stacks}

	Stacks are a very simple data structure, in some ways limited, but also
	very fast for some of simple operations.

	A basic stack provides push and pop operations to store and receive the
	topmost item, respectively. This is easy to implement as a
	(possibly single-direction) linked list.

	\textbf{Example}

\begin{verbatim}
	#include <stdio.h>;

	struct page *pagestk;

	struct page {
	    void        *addr;
	    struct page *next;
	};

	#define pushpage(page) ((page)->next = pagestk,
	                        pagestk = (page))
	#define poppage(tmp)   ((tmp) = pagestk, \
	                        pagestk
			            = ((tmp)
				       ? (tmp)->next
				       : NULL),
			        (tmp))

	int
	main(int argc,
	     char *argv)
	{
	    struct page *tmp;
	    struct page  pg1 = { &pg1, NULL };
	    struct page  pg2 = { &pg2, NULL };

	    fprintf(stderr, "push\t%p\n", &pg1);
	    pushpage(&pg1);
    	    fprintf(stderr, "push\t%p\n", &pg2);
    	    pushpage(&pg2);
    	    fprintf(stderr, "pop\t%p\n", poppage(tmp));
    	    fprintf(stderr, "pop\t%p\n", poppage(tmp));
    	    fprintf(stderr, "pop\t%p\n", poppage(tmp));

    	    exit(0);
	}
\end{verbatim}

\section{Tricks}

\begin{verbatim}
	/* zero page, 32-bit version. */

	void
	pzero(void *ptr)
	{
	    unsigned long  zero = 0;        /* do not embed big constant into loop. */
    	    unsigned long  cnt = PGSZ >> 4; /* loop count. */
    	    unsigned long *ulptr = ptr;     /* unsigned long at a time. */

    	    /*
     	     * - using small constants as pointer offsets is faster than variables
     	     *   for me.
     	     *   -vendu
     	     */
	    while (cnt--) {
                ulptr[0] = ulptr[1] = ulptr[2] = ulptr[3] = zero;
        	 ulptr += 4;
	    }

	    return;
	}
\end{verbatim}

\chapter{Tools of the Trade}

\section{Software}

	\textbf{http://www.valgrind.org/}

	great memory debugger (and cache simulator)

	\textbf{http://icl.cs.utk.edu/papi/software/}

	PAPI performance monitor

\section{Suggested Reading}

	\textbf{http://www.plantation-productions.com/Webster/www.writegreatcode.com/index.html}

	Hyde, Randall; Write Great Code book series

	\textbf{http://www.hackersdelight.org/}

	Warren, Henry S Jr.; Hacker's Delight

	\textbf{http://www.amazon.com/Code-Optimization-Effective-Memory-Usage/dp/1931769249}

	Kaspersky, Kris; Code Optimization: Effective Memory Usage

	\textbf{http://ourworld.compuserve.com/homepages/rbooth/}

	Booth, Rick; Inner Loops

\newpage

\section{Links}

	\textbf{http://aggregate.org/MAGIC/}

	Aggregate Magic Algorithms

	\textbf{http://graphics.stanford.edu/~seander/bithacks.html}

	Stanford Bit Twiddling Hacks

	\textbf{http://www.inwap.com/pdp10/hbaker/hakmem/hakmem.html}

	MIT HAKMEM

\chapter{System Calls}

\chapter{IA-32 and X86-64 Implementations}

\section{Calling Conventions}

\subsection{Stack Frames}

\section{Hardware Interface}

\subsection{Memory Segments}

\subsection{Traps}

\subsection{Trap Frames}


