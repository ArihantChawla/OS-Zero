\documentclass[llpt, twoside, a4paper]{book}
\usepackage[T1]{fontenc}
\usepackage{parskip}
%\usepackage{listings}
\usepackage{fancyvrb}
%\lstset{language=C, showspaces=false, breaklines=false}

\begin{document}

\title{\LARGE{Wizard Pseudo Machine} \\*
       \large{Volume One, Programmer's Guide, revision 0.0.8}}
\author{Tuomo Petteri Ven�l�inen}
\date{\today}
\maketitle

\setcounter{secnumdepth}{3}
\setcounter{tocdepth}{4}
\setlength{\parindent}{0cm}

\tableofcontents

\part{Preface}

\chapter{Background and Future}

	Wizard Pseudo Machine project started as an attempt to create a tool for
	educational purposes. The machine is a virtual processor with custom
	instruction set that is close to existing ones; goals include making
	this instruction set simple, relatively complete, and be useful for
	C language operations in case someone wants to create a C compiler or
	code generator for an existing one.

	At the time of this writing, the virtual assembler is in good shape.
	The assembler provides not only .include to place other files into the
	stream verbatim, but also .import to 'link' with other assembly files
	with access to their global symbols.

	The project is by no means complete yet. I wish for useful libraries for
	audio and graphics, a useful stock 'standard' library, and lots of other
	things to happen in the near future. :)

\chapter{Changes}

	\textbf{0.0.8}

\begin{itemize}
	\item{added .asciz}
	\item{added new instruction \textbf{hook} to invoke high level services;
	this lets us run them with native code in the virtual machine}
\end{itemize}

	\textbf{0.0.7}

\begin{itemize}
	\item{adding more content}
	\item{cleaning up}
\end{itemize}

	\textbf{0.0.6}

\begin{itemize}
	\item{still fixing typoes}
	\item{added a new \textbf{Architecture} Chapter}
\end{itemize}

	\textbf{0.0.5}

\begin{itemize}
	\item{Added text on threading as well as a couple of assembly examples}
\end{itemize}

	\textbf{0.0.4}

\begin{itemize}
	\item{changed the title, made a couple of mistakes there; the book is
	now correctly called \textbf{Wizard Pseudo Machine}}
\end{itemize}

	\textbf{0.0.3}

\begin{itemize}
	\item{added new subsection Opcode Format}
\end{itemize}

	\textbf{0.0.2}

\begin{itemize}
	\item{reorganised some assembly sections; added .space, .long, .short, and .byte}
\end{itemize}

	\textbf{0.0.1}

\begin{itemize}
	\item{changed the assembler to use .include and .import instead of \#include and \#import}
	\item{changed the term 'argument' to 'operand' in many places in this booklet}
\end{itemize}

\part{Pseudo Machine}

\chapter{Architecture}

	The pseudo machine supports flat 4-gigabyte address space, of which some
	is mapped for interrupt vector, interrupt handlers, thread and
	interrupt stacks, graphics, and other purposes.

	Native word size is \textbf{32-bit}. Words are little endian, i.e.
	lowest byte first.

	There exists an instruction, \textbf{thr}, to start executing new
	threads at the desired locations in memory.

	We follow the \textbf{von Neumann architecture}, so we basically have
	3 abstraction; CPU, memory, and input/output.

	The machine is a purposefully \textbf{RISC}-like load-store approach,
	meaning there is only a single load-store instruction (\textbf{mov})
	that deals with memory addressed operands.

\section{Memory Map}

	\textbf{Notes}

\begin{itemize}
	\item{the VM's memory size is currently specified as \textbf{MEMSIZE}}
	\item{thread stacks live at \textbf{MEMSIZE - thrid * THRSTKSIZE}}
\end{itemize}

\begin{tabular}{ | l | l | l | }
	\hline
	Address         & Purpose          & Brief \\
	\hline
	0               & interrupt vector & interrupt handler function pointers \\
	4096            & keyboard buffer  & keyboard input queue \\
	8192            & text segment     & application program code (read-execute) \\
	8192 + TEXTSIZE & data segment     & program data (read-write) \\
	DATA + DATASIZE & BSS segment      & uninitialised data (runtime-allocated and zeroed) \\
	MEMSIZE - 3.5 G & dynamic segment  & free space for slab allocator \\
	3.5 gigabytes   & graphics         & 32-bit ARGB-format draw buffer \\
	\hline
\end{tabular}

\chapter{Instruction Set}

\section{Machine Operations}

	The following is a C-code snippet listing machine instructions and their
	IDs in opcodes.

\begin{verbatim}
        #define OPNOT      0x01 // 2's complement
        #define OPAND      0x02 // logical AND
        #define OPOR       0x03 // logical OR
        #define OPXOR      0x04 // logical exclusive OR
        #define OPSHL      0x05 // shift left (fill with zero)
        #define OPSHR      0x06 // arithmetic shift right (fill with sign)
        #define OPSHRL     0x07 // logical shift right (fill with zero)
        #define OPROR      0x08 // rotate right
        #define OPROL      0x09 // rotate left
        #define OPINC      0x0a // increment by one
        #define OPDEC      0x0b // decrement by one
        #define OPADD      0x0c // addition
        #define OPSUB      0x0d // subtraction
        #define OPCMP      0x0e // compare
        #define OPMUL      0x0f // multiplication
        #define OPDIV      0x10 // division
        #define OPMOD      0x11 // modulus
        #define OPBZ       0x12 // branch if zero
        #define OPBNZ      0x13 // branch if not zero
        #define OPBLT      0x14 // branch if less than
        #define OPBLE      0x15 // branch if less than or equal to
        #define OPBGT      0x16 // branch if greater than
        #define OPBGE      0x17 // branch if greater than or equal to
        #define OPBO       0x18 // branch if overflow
        #define OPBNO      0x19 // branch if no overflow
        #define OPBC       0x1a // branch if carry
        #define OPBNC      0x1b // branch if no carry
        #define OPPOP      0x1c // pop from stack
        #define OPPUSH     0x1d // push to stack
        #define OPMOV      0x1e // load/store 32-bit longword
        #define OPMOVB     0x1f // load/store 8-bit byte
        #define OPMOVW     0x20 // load/store 16-bit word
        #define OPJMP      0x21 // jump to given address
        #define OPCALL     0x22 // call subroutine
        #define OPENTER    0x23 // subroutine prologue
        #define OPLEAVE    0x24 // subroutine epilogue
        #define OPRET      0x25 // return from subroutine
        #define OPLMSW     0x26 // load machine status word
        #define OPSMSW     0x27 // store machine status word
        #define OPRESET    0x28 // reset into well-known state
        #define OPNOP      0x29 // dummy operation
        #define OPHLT      0x2a // halt execution
        #define OPBRK      0x2b // breakpoint
        #define OPTRAP     0x2c // trigger a trap (software interrupt)
        #define OPIRET     0x2d // return from interrupt handler
        #define OPTHR      0x2e // start new thread at given address
        #define OPCMPSWAP  0x2f // atomic compare and swap
        #define OPINB      0x30 // read 8-bit byte from port
        #define OPOUTB     0x31 // write 8-bit byte to port
        #define OPINW      0x32 // read 16-bit word
        #define OPOUTW     0x33 // write 16-bit word
        #define OPINL      0x34 // read 32-bit long
        #define OPOUTL     0x35 // write 32-bit long
\end{verbatim}

\section{Reference}

\subsection{Opcode Format}

	The following C structure is what the stock assembler uses for opcode
	output.

	\textbf{Opcode Structure}

\begin{verbatim}
	struct wpmopcode {
	    unsigned inst     : 8;	// instruction ID
	    unsigned unit     : 2;	// unit ID
	    unsigned arg1t    : 3;	// argument #1 type
	    unsigned arg2t    : 3;      // argument #2 type
	    unsigned reg1     : 6;	// register #1 ID + addressing flags
	    unsigned reg2     : 6;	// register #2 ID + addressing flags
	    unsigned size     : 2;      // 1..3, shift count
	    unsigned res      : 2;      // reserved
	    int32_t  args[2];
	} __attribute__ ((__packed__));
\end{verbatim}

	\textbf{Notes}

\begin{itemize}
	\item{\textbf{inst} is the instruction ID; 0 is invalid}
	\item{\textbf{unit} is a future unit ID; ALU, FPU, VPU (SIMD), GPU?}
	\item{\textbf{reg1 and reg2} are source and destination register IDs}
	\item{\textbf{operation size} can be calculated as \textbf{op->size << 2}}
	\item{\textbf{res}-bits are reserved for future extensions}
	\item{\textbf{args} contains 0, 1, or 2 32-byte addresses or values}
\end{itemize}

\subsection{Instruction Set}

	\textbf{Operand Types}

\begin{itemize}
	\item{\textbf{r} stands for register operand}
	\item{\textbf{i} stands for immediate operand value}
	\item{\textbf{a} stands for immediate direct address operand}
	\item{\textbf{p} stands for indirect address operand}
	\item{\textbf{n} stands for indexed address operand}
	\item{\textbf{m} stands for all of \textbf{a, i, and n}}
\end{itemize}

	\textbf{Notes}

\begin{itemize}
	\item{C language doesn't specify whether right shifts are arithmetic or
	logical}
	\item{Arithmetic right shift fills leftmost 'new' bits with the sign
	bit, logical shift fills with zero; left shifts are always fill
	rightmost bits with zero}
\end{itemize}

	\textbf{Instructions}

	Below, I will list machine instructions and illustrate their relation
	to C.

	\textbf{Notes}

\begin{itemize}
	\item{the \textbf{inb()} and other functions dealing with I/O are
	usually declared through \textbf{<sys/io.h>}}
\end{itemize}

\newpage

\begin{tabular}{ | l | l | l | l | }
	\hline
	C Operation & Instruction & Operands       & Brief \\
	\hline
	~           & not         & r dest          & reverse all bits \\
	\&          & and         & r/i src, r dest & logical AND \\
	|           & or          & r/i src, r dest & logical OR \\
	\textasciicircum & xor    & r/i src, r dest & logical exclusive OR \\
	<<          & shl         & r/i cnt, r dest & shift left by count \\
	>>          & shr         & r/i cnt, r dest & arithmetic shift right \\
		    & shrl        & r/i cnt, r dest & logical shift right \\
	N/A         & ror         & r/i cnt, r dest & rotate right by count \\
	N/A         & rol         & r/i cnt, r dest & rotate left by count \\
	++          & inc         & r dest          & increment by one \\
	--          & dec         & r dest          & decrement by one \\
	+           & add         & r/i cnt, r dest & addition \\
	-           & sub         & r/i cnt, r dest & subtraction \\
	==, != etc. & cmp         & r/i src, r dest & comparison; sets MSW-flags \\
	**          & mul         & r/i src, r dest & multiplication \\
	/           & div         & r/i src, r dest & division \\
	\%          & mod         & r/i src, r dest & modulus \\
	==, !       & bz          & none            & branch if zero \\
	!=, (val)   & bnz         & none            & branch if not zero \\
	<           & blt         & none            & branch if less than \\
	<=          & ble         & none            & branch if less than or equal \\
	>           & bgt         & none            & branch if greater than \\
	>=          & bge         & none            & branch if greater than or equal \\
	N/A         & bo          & none            & branch if overflow \\
	N/A         & bno         & none            & branch if no overflow \\
	N/A         & bc          & none            & branch if carry \\
	N/A         & bnc         & none            & branch if no carry \\
	dest = *sp++ & pop         & r dest          & pop from stack \\
	*--sp = src & push        & r src           & push onto stack \\
	dest = src  & mov         & r/i/m src, r/m dest & load/store longword (32-bit) \\
	dest = src  & movb        & r/i/m src, r/m dest & load/store byte (8-bit) \\
	dest = src  & movw        & r/i/m src, r/m dest & load/store word (16-bit) \\
	N/A         & jmp         & r/m dest        & continue execution at dest \\
	N/A         & call        & a/p dest        & call subroutine \\
	N/A         & enter       & none            & subroutine prologue \\
	N/A         & leave       & none            & subroutine epilogue \\
	N/A         & ret         & none            & return from subroutine \\
	N/A         & lmsw        & r/i dest        & load machine status word \\
	N/A         & smsw        & r/i src         & store machine status word \\
	N/A         & reset       & none            & reset machine \\
	N/A         & nop         & none            & no operation \\
	N/A         & hlt         & none            & halt machine \\
	N/A         & brk         & none            & breakpoint \\
	N/A         & trap        & r/i src         & trigger software interrupt \\
	N/A         & cli         & none            & disable interrupts \\
	N/A         & sti         & none            & enable interrupts \\
	N/A         & iret        & none            & return from interrupt handler \\
	N/A         & thr         & r/i dest        & start thread at dest \\
	N/A         & cmpswap     & r/i src, m dest & atomic compare and swap \\
	inb()       & inb         & r/i src         & read byte (8-bit) \\
	outb()      & outb        & r/i dest        & write byte (8-bit) \\
	inw()       & inw         & r/i src         & read word (16-bit) \\
	outw()      & outw        & r/i dest        & write word (16-bit) \\
        inl()       & inl         & r/i src         & read longword (32-bit) \\
	outl()      & outl        & r/i dest        & write longword (32-bit) \\
	\hline
\end{tabular}

\chapter{Assembly}

\section{Syntax}

	\textbf{AT\&T}

	We use so-called AT\&T-syntax assembly. Perhaps the most notorious
	difference from Intel-syntax is the operand order; AT\&T lists the
	source operand first, destination second, whereas Intel syntax does it
	vice versa.

	\textbf{Symbol Names}

	Label names must start with an underscore or a letter; after that, the
	name may contain underscores, letters, and digits. Label names end with
	a ':', so like

\begin{verbatim}
	value:	.long 0xb4b5b6b7
\end{verbatim}

	would declare a longword value at the address of \textbf{value}.

	\textbf{Instructions}

	The instruction operand order is source first, then destination. For
	example,

\begin{verbatim}
	mov	8(%r0), val
\end{verbatim}

	would store the value from address \textbf{r0 + 8} to the address of
	the label \textbf{val}.

	\textbf{Operands}

	Register operand names are prefixed with a '\textbf{\%}. Immediate
	constants and direct addresses are prefixed with a \'textbf{\$}'. Label
	addresses are refered to as their names without prefixes.

	The assembler supports simple preprocessing (of constant-value
	expressions), so it is possible to do things such as

\begin{verbatim}
	mov	$(0x01 | 0x02), %r1
\end{verbatim}

	\textbf{Registers}

	Register names are prefixed with '\%'; there are 16 registers r0..r15.
	For example,

\begin{verbatim}
	add	%r0, %r1
\end{verbatim}

	would add the longword in r0 to r1.

	\textbf{Direct Addressing}

	Direct addressing takes the syntax

\begin{verbatim}
	mov	val, %r0
\end{verbatim}

	which moves the longword at \textbf{address val} into r0.

	\textbf{Indexed Addressing}

	Indexed	addressing takes the syntax

\begin{verbatim}
	mov	4(%r0), %r1
\end{verbatim}

	where 4 is an integral constant offset and r0 is a register name. In
	short, this would store the value at the address \textbf{r0 + 4} into
	r1.

	\textbf{Indirect Addressing}

	Indirect addresses are indicated with a \textbf{'*'}, so

\begin{verbatim}
	mov	*%r0, %r1
\end{verbatim}

	would store the value from the \textbf{address in the register r0}
	into register r1, whereas

\begin{verbatim}
	mov	*val, %r0
\end{verbatim}

	would move the value \textbf{pointed to by val} into r0.

	Note that the first example above was functionally equivalent with

\begin{verbatim}
	mov	(%r0), %r1
\end{verbatim}

	\textbf{Immediate Addressing}

	Immediate addressing takes the syntax

\begin{verbatim}
	mov	$str, %r0
\end{verbatim}

	which would store the \textbf{address of str} into r0.

\section{Assembler Directives}

\subsection{Input Directives}

\subsubsection{.include}

	The \textbf{.include} directive takes the syntax

\begin{verbatim}
	.include <file.asm>
\end{verbatim}

	to insert file.asm into the translation stream verbatim.

\subsubsection{.import}

	The \textbf{.import} directive takes the syntax

\begin{verbatim}
	.import <file.asm>
\end{verbatim}

	or

\begin{verbatim}
	.import <file.obj>
\end{verbatim}

	to import foreign assembly or object files into the stream.
	\textbf{Note} that only symbols declared with \textbf{.globl} will be
	made globally visible to avoid namespace pollution.

\subsection{Link Directives}

\subsubsection{.org}

	The \textbf{.org} directive takes a single argument and sets the linker
	location address to the given value.

\subsubsection{.space}

	The \textbf{.space} directive takes a single argument and advances the
	link location address by the given value.

\subsubsection{.align}

	The \textbf{.align} directive takes a single argument and aligns the
	next label, data, or instruction to a boundary of the given size.

\subsubsection{.globl}

	The \textbf{.globl} directive takes one or several symbol names
	arguments and declares the symbols to have global visibility (linkage).

\subsection{Data Directives}

\subsubsection{.long}

	\textbf{.long} takes any number of arguments and declares in-memory
	32-bit entities.

\subsubsection{.byte}

	\textbf{.byte} takes any number of arguments and declares in-memory
	8-bit entities.

\subsubsection{.short}

	\textbf{.short} takes any number of arguments and declares in-memory
	16-bit entities.

\subsubsection{.asciz}

	\textbf{.asciz} takes a C-style string argument of characters enclosed
	within double quotes ('"'). Escape sequences '$\backslash$n' (newline),
	'$\backslash$t' (tabulator), and '$\backslash$r' (carriage return) are supported.

\subsection{Preprocessor Directives}

\subsubsection{.define}

	\textbf{.define} lets one declare symbolic names for constant (numeric)
	values. For example, if you have

	\textbf{<hook.def>}

\begin{verbatim}
        .define PZERO  0
        .define PALLOC 1
        .define PFREE  2
\end{verbatim}

	you can then use the symbolic names like

\begin{verbatim}
        .include <hook.def>
        .import <bzero.asm>

        memalloc:	
                mov     $16384, %r0
                hook    $PALLOC
                mov     %r0, ptr
                ret

        memzero:
                mov     ptr, %r0
                mov     $4096, %r1
                hook    $PZERO
                ret

        memfree:
                mov     ptr, %r0
                hook    $PFREE
                ret

        _start:
                call    memalloc
                call    memzero
                call    memfree
                hlt

        ptr:    .long   0x00000000
\end{verbatim}

\section{Input and Output}

	The pseudo machine uses some predefined ports for keyboard and console
	I/O. The currently predefined ports are

\begin{tabular}{ | l | l | l | }
	\hline
	Port & Use            & Notes \\
	\hline
	0x00 & keyboard input & interrupt-driven \\
	0x01 & console output & byte stream \\
	0x02 & error output   & directed to console by default \\
	\hline
\end{tabular}

\subsection{Simple Program}

	The following code snippet prints the string \"hello\" + a newline to
	the console. Note that the string is saved using the standard C
	convention of NUL-character termination.

\begin{verbatim}
        msg:        .asciz	"hello\n"

        .align      4
        
        _start:        
                mov        $msg, %r0
                movb       *%r0, %r1
                mov        $0x01, %r2
                cmp        $0x00, %r1
                bz         done
        loop:
                inc        %r0
                outb       %r1, %r2
                movb       *%r0, %r1
                cmp        $0x00, %r1
                bnz        loop
        done:
                hlt
\end{verbatim}

\section{Threads}

	Wizard Pseudo Machine supports hardware threads with the \textbf{thr}
	instruction. It takes a single argument, which specifies the new
	execution start address; function arguments should be passed in
	registers.

\subsection{Example Program}

	The following piece of code shows simple utilisation of threads.

\begin{verbatim}
        .import <bzero.asm>

        memzero:
                mov        $65536, %r0       // address
                mov        $4096, %r1        // length
                call       bzero
                hlt

        _start:
                thr        $memzero
                hlt
\end{verbatim}

\section{Hooks}

	Hooks exist to provide system services. Hooks invoke native code in the
	virtual machine to do things such as manage memory and I/O.

\subsection{Pre-Defined Hooks}

\begin{tabular}{ | l | l | l | }
	\hline
	Number & Name    & Purpose \\
	\hline
	0x00   & PZERO  & zero given number of pages at given address \\
	0x01   & PALLOC & allocate given number of bytes from dynamic segment \\
	0x01   & PFREE  & free region at given address \\
	\hline
\end{tabular}

\subsection{Hook Interface}

	Hook \textbf{parameters} are passed \textbf{in registers}. Hook
	\textbf{return value} is stored \textbf{in r0}. Here is the current
	interface definition.

\begin{itemize}
	\item{PZERO takes two arguments; destination address in r0, and region
	size (in bytes) in r1. PZERO returns nothing.}
	\item{PALLOC takes one argument; allocation size in r0. PALLOC returns
	allocated address or zero on failure.}
	\item{PFREE takes one argument; allocation address in r0. PFREE returns
	nothing.}
\end{itemize}

\subsection{Example Program}

	The following programs uses hooks to accomplish 3 tasks: allocate 16384
	bytes of memory, zero it, and finally free it. In reality this alone is
	useless, but it serves as an example.

\begin{verbatim}
        .import <bzero.asm>

        alloc:	
	        mov     $16384, %r0
	        hook    $1
	        mov     %r0, ptr
	        ret

        zero:
	        mov     ptr, %r0
	        mov     $16384, %r1
	        hook    $0
	        ret

        free:
	        mov	ptr, %r0
	        hook	$2
	        ret

        _start:
	        call    alloc
	        call    zero
	        call    free
	        hlt

        ptr:    .long    0x00000000

        _foo:   .space   4096, 0xff


\end{verbatim}

\section{Interrupts}

	Software- and CPU-generated interrupts are often refered to as
	\textbf{traps}. I call those and hardware-generated
	\textbf{interrupt requests} interrupts, collectively.

\subsection{Break Points}

	The \textbf{brk} instruction triggers a breakpoint interrupt. The
	default action is to print a stack trace and continue execution.

	The \textbf{use} of brk is simple; just use the zero-operand
	instruction in your assembly file:

\begin{verbatim}
	brk	; trigger breakpoint
\end{verbatim}

\subsection{Interrupt Interface}

	The lowest page (4096 bytes) in virtual machine address space contains
	the \textbf{interrupt vector}, i.e. a table of interrupt handler
	addresses to trigger them.

	Interrupt handler invokations only push the \textbf{program counter}
	and \textbf{old frame pointer}, so you need to reserve the registers
	you use manually. This is so interrupts could be as little overhead as
	possible to handle.

\subsection{Keyboard Input}

	In order to read keyboard input without polling, we need to hook the
	\textbf{interrupt 0}. This is done in two code modules; an interrup
	handler as well as other support code.

	I will illustrate the interrupt handler first.

\subsubsection{Keyboard Interrupt Handler}

	\textbf{TODO: example interrupt handler}

\subsubsection{Keyboard Support Code}

	\textbf{TODO: queue keypresses in 16-bit values; 32-bit if full Unicode requested}.

\end{document}

